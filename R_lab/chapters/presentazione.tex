\chapter{Presentazione del Corso}


\section{Argomenti trattati}
\label{sec:presentazione}

Queste dispense si basano (Capitoli~\ref{cap:installareR}~-~\ref{cap:dati}), sul corso \textbf{R for Reproducible Scientific Analysis}, sviluppato dal collettivo \textbf{Software Carpentries}. Per motivi di tempo, ci concentreremo solo sulle funzioni che permettono semplici manipolazioni e analisi dei dati. 

\noindent I Capitoli~\ref{cap:descrittiva} e~\ref{cap:inferenziale} rappresentano del materiale originale, sviluppato appositamente per questo corso dal docente (Alessia Visconti, \href{mailto:alessia.visconti@unito.it}{alessia.visconti@unito.it}) e sono legati agli argomenti introdotti in modo teorico nella prima parte del corso di Statistica Medica. 

\noindent Non affronteremo nessun elemento di programmazione in R, che \`e al di fuori dello scopo di questo corso e neppure concetti avanzati di statistica. Per approfondimenti si rimanda alla Sezione~\ref{sec:corsi_carpentries}. 


% \secton{Perch\'e fare un laboratorio di R?}
%
% FIXME

\section{Approfondimenti}
\label{sec:corsi_carpentries}

Per chi volesse approfondire gli argomenti trattati, si rimanda al corso citato in Sezione~\ref{sec:presentazione}, ma anche al principale corso in programmazione R sviluppato dalle Carpentries, elencati in seguito. Entrambi i corsi sono pensati per persone che non hanno alcuna esperienza con la programmazione e/o l'analisi dati e alcuni argomenti sono ripetuti tra i due corsi:

\begin{myitemize}
	\item R for Reproducible Scientific Analysis \\ \url{https://swcarpentry.github.io/r-novice-gapminder}
	\item Programming with R \\ \url{https://swcarpentry.github.io/r-novice-inflammation}
\end{myitemize}

\noindent Esistono anche corsi in R specifici per alcune discipline, sempre sviluppati dalle Carpentries, che potete trovare al seguente indirizzo: \url{https://datacarpentry.org/lessons/}, insieme a molti altri corsi.

\noindent Se vi appassionate all'argomento, vi consiglio anche i seguenti testi, tutti disponibili gratuitamente online agli indirizzi indicati: 

\begin{myitemize}
	\item \textbf{R for Data Science} \\ \url{https://r4ds.had.co.nz/}
	\item \textbf{Learning Statistics with R}\footnote{L'autrice di questo libro, Danielle Navarro, si occupa anche di Generative Art e ha un blog molto interessante su R e Data Science. Trovate altro materiale che ha sviluppato nel suo sito: \url{https://djnavarro.net/}} \\ \url{https://learningstatisticswithr.com/lsr-0.6.pdf}
	\item \textbf{ggplot2: Elegant Graphics for Data Analysis} \\ \url{https://ggplot2-book.org/}
\end{myitemize}
	
\section{Ringraziamenti}

Vorrei ringraziare le Carpentries, non solo per il materiale che forniscono gratuitamente, ma anche per avermi reso un docente migliore, sotto molti punti fui vista. Ringrazio anche il professor Roberto Esposito, per avermi permesso di utilizzare il template \LaTeX ~che ha ideato -- le variazioni che lo rendono meno accattivante graficamente sono tutte mie. 


\noindent Per ultimo, un grosso grazie agli studenti del corso di laurea in Tecniche della Riabilitazione Psichiatrica, a.a. 2024/25, per essere stati i beta-tester di questo corso.


\section{Licenza}


This work is licensed under a Creative Commons Attribution-NonCommercial-ShareAlike 4.0 International License (cc-by-nc-sa).
See \url{http://creativecommons.org/licenses/by-nc-sa/4.0/} for details. 

\vspace{0.5cm}

\begin{figure}[h!]
 \centering
  \includegraphics[width=0.25\textwidth]{images/CC BY-NC-SA 4.0.png}
\end{figure}