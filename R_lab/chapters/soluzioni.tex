\chapter{Soluzioni agli esercizi}

\section{Capitolo 3}

\begin{solution}{ex2.1}

\noindent I seguenti nomi di variabile sono validi in R:

\begin{lstlisting}[style=Rstyle]
min_height
max.height
MaxLength
celsius2kelvin
\end{lstlisting}

\noindent anche questo nome \`e valido, la sua particolarit\`a \`e che Crea una variabile nascosta:

\begin{lstlisting}[style=Rstyle]
.mass
\end{lstlisting}

\noindent mentre i seguenti nomi di variabile non sono validi:

\begin{lstlisting}[style=Rstyle]
_age
min-length
2widths
\end{lstlisting}


\end{solution}

\vspace{0.5cm}

\begin{solution}{ex2.2}

\begin{lstlisting}[style=Rstyle]
# Assegna a mass il valore 50.5
x <- 50.5    

# Assegna a age il valore 122
y <- 122     

# Moltiplica il valore contenuto in x per 2 e lo ri-assegna 
x <- x * 2   

# Sottrae 20 al valore contenuto in y e lo ri-assegna 
y <- y - 20  
	
x
[1] 101
y
[1] 102
\end{lstlisting}

\end{solution}

\vspace{0.5cm}

\begin{solution}{ex2.3}

\noindent Due dei comandi che possono essere usati sono:

\begin{lstlisting}[style=Rstyle]
x > y	
[1] FALSE

x >= y
[1] FALSE
\end{lstlisting}

\end{solution}

\vspace{0.5cm}

\begin{solution}{ex2.4}

\begin{lstlisting}[style=Rstyle]
# Crea la variabile mass e le assegna il valore 2.2
mass <- 2.2   

# Crea la variabile age e le assegna il valore 15
age <- 15     

# Per verificare che esistano posso visualizzare 
# tutte le variabili nell'ambiente di lavoro
ls()
[1] "age" "mass"

# Per verificare che siano state inizializzate correttamente 
# posso stamparle a video 
mass 
[1] 2.2
age 
[1] 15

#Rimuove le due variabili
rm(mass, age)  

# Per verificare che siano state rimosse posso visualizzare 
# tutte le variabili nell'ambiente di lavoro
ls()
character(0)
\end{lstlisting}

\end{solution}


\vspace{0.5cm}

\begin{solution}{ex2.5}

\begin{lstlisting}[style=Rstyle]
# Carico un pacchetto R specificato nella sessione di lavoro
library(ggplot2) 
\end{lstlisting}


\begin{lstlisting}[style=Rstyle]
# Installo il pacchetto specificato (attenzione alle virgolette)
install.packages("ggrepel")  
\end{lstlisting}	

\end{solution}	


\newpage

\section{Capitolo 4}

\begin{solution}{ex3.1}

\begin{lstlisting}[style=Rstyle]
df$country
 [1] "Benin"      "Greece"     "Tanzania"  
 [4] "Estonia"    "Russia"     "Syria"     
 [7] "Zambia"     "Vietnam"    "Liberia"   
[10] "Mozambique"
\end{lstlisting}

\noindent Ritorna la prima colonna di un data frame, usandone il nome, come un vettore di caratteri (tipo: \lsin{character}) ed \`e equivalente a:

\begin{lstlisting}[style=Rstyle]
df[, 1]
 [1] "Benin"      "Greece"     "Tanzania"  
 [4] "Estonia"    "Russia"     "Syria"     
 [7] "Zambia"     "Vietnam"    "Liberia"   
[10] "Mozambique"
\end{lstlisting}
%
che usa l'indice di posizione della colonna. Analogamente,

\begin{lstlisting}[style=Rstyle]
df[1, ]
  country world_region income_group_2017
1   Benin       africa        Low income
  happiness_score_2011 gov_health_spending_percent
1                 38.7                         9.6
\end{lstlisting}
%
ritorna non la prima colonna, ma la prima riga. Invece

\begin{lstlisting}[style=Rstyle]
df[1, 1]
[1] "Benin"
\end{lstlisting}
%
ritorna il contenuto della cella in posizione \lsin{<1,1>}.

\begin{lstlisting}[style=Rstyle]
df["country"]
      country
1       Benin
2      Greece
3    Tanzania
4     Estonia
5      Russia
6       Syria
7      Zambia
8     Vietnam
9     Liberia
\end{lstlisting}
%
invece crea un nuovo data frame, andando a tagliare una "fetta" di quello originale.

\end{solution}

\vspace{0.5cm}

\begin{solution}{ex3.2}

\begin{lstlisting}[style=Rstyle]
str(cats)
'data.frame' : 4 obs. of  4 variables:
 $ coat      : chr  "calico" "black" "tabby" "white"
 $ weight    : chr  "2.1" "5" "3.2" "3.3"
 $ playful   : chr  "1" "0" "1" "1"
 $ age       : chr  "2" "3" "5" "9"
\end{lstlisting}
%
Tutte le colonne sono state trasformate in \lsin{character}. Questo \`e successo perch\'e un vettore pu\`o includere un solo tipo di dati, e il meno restrittivo \`e  \lsin{character}.

\end{solution}

\vspace{0.5cm}

\begin{solution}{ex3.3}

\noindent Punto 1: Creare un data frame chiamato \lsin{classe} che contenga le seguenti informazioni su noi stessi: \lsin{<Nome, Cognome, Giorno di Nascita>}

\begin{lstlisting}[style=Rstyle]
classe <- data.frame(Nome = "Alessia",
                     Cognome = "Visconti",
                     GiornoNascita  = 4)
\end{lstlisting}
%
Soluzione alternativa:

\begin{lstlisting}[style=Rstyle]
classe <- data.frame(Nome = c("Alessia"),
                     Cognome = c("Visconti"),
                     GiornoNascita  = c(4))
\end{lstlisting}

\vspace{0.2cm}
\noindent Punto 2: Aggiungere delle righe che contengano le stesse informazioni per due dei vicini:

\begin{lstlisting}[style=Rstyle]
vicino1 <- list("Tizio", "Caio", "12")
vicino2 <- list("Pinco", "Pallino", "20")
classe <- rbind(classe, vicino1)
classe <- rbind(classe, vicino2)
\end{lstlisting}
%
Soluzione alternativa:

\begin{lstlisting}[style=Rstyle]
vicino1 <- list("Tizio", "Caio", "12")
vicino2 <- list("Pinco", "Pallino", "20")
classe <- rbind(classe, vicino1, vicino2)
\end{lstlisting}
%
Soluzione alternativa:

\begin{lstlisting}[style=Rstyle]
classe <- rbind(classe, list("Tizio", "Caio", "12"),
                        list("Pinco", "Pallino", "20"))
\end{lstlisting}
%
Soluzione alternativa:

\begin{lstlisting}[style=Rstyle]
vicini <- data.frame(Nome = c("Tizio", "Pinco"),
                     Cognome = c("Caio", "Pallino"),
                     GiornoNascita  = c(12, 20))
classe <- rbind(classe, vicini))
\end{lstlisting}
%
Altre soluzioni sono possibili.

\vspace{0.2cm}
\noindent Punto 3:  Aggiungere una colonna con la mostra risposta (\lsin{TRUE/FALSE}) alla domanda \emph{Mi servirebbe una pausa?}

\begin{lstlisting}[style=Rstyle]
risposta <- c("FALSE", "TRUE", "FALSE")
classe <- cbind(classe, risposta)
\end{lstlisting}
%
Soluzione alternativa:
\begin{lstlisting}[style=Rstyle]
classe <- cbind(classe, risposta=c("FALSE", "TRUE", "FALSE"))
\end{lstlisting}
%
Altre soluzioni sono possibili.

\end{solution}

\vspace{0.5cm}

\begin{solution}{ex3.4}

\noindent Comando 1:
\begin{lstlisting}[style=Rstyle]
x[2:4]
  b   c   d
6.2 7.1 4.8
\end{lstlisting}
%
\noindent Comando 2:
\begin{lstlisting}[style=Rstyle]
x[-c(1,5)]
  b   c   d
6.2 7.1 4.8
\end{lstlisting}
%
\noindent Comando 3:
\begin{lstlisting}[style=Rstyle]
x[c(2,3,4)]
  b   c   d
6.2 7.1 4.8
\end{lstlisting}

\end{solution}

\vspace{0.5cm}

\begin{solution}{ex3.5}

\begin{lstlisting}[style=Rstyle]
x[4 < x & x < 7]

  a   b   d
5.4 6.2 4.8 
\end{lstlisting}

\end{solution}

\vspace{0.5cm}

\begin{solution}{ex3.6}

\noindent Punto 1: estrarre tutti i valori per paesi europei

\begin{lstlisting}[style=Rstyle]
df[df$world_region == "europe", ]

  country world_region income_group_2017
2  Greece       europe       High income
4 Estonia       europe       High income
5  Russia       europe     Middle income
  happiness_score_2011 gov_health_spending_percent
2                 53.7                       12.10
4                 54.9                       11.70
5                 53.9                        8.04
\end{lstlisting}
%
\noindent Punto 2: estrarre tutti i valori per i paesi non nel gruppo "Middle income"

\begin{lstlisting}[style=Rstyle]
df[df$income_group_2017 != "Middle income", ]
      country world_region income_group_2017
1       Benin       africa        Low income
2      Greece       europe       High income
3    Tanzania       africa        Low income
4     Estonia       europe       High income
9     Liberia       africa        Low income
10 Mozambique       africa        Low income
   happiness_score_2011 gov_health_spending_percent
1                  38.7                         9.6
2                  53.7                        12.1
3                  40.7                        13.8
4                  54.9                        11.7
9                    NA                        11.1
10                 49.7                        12.2
\end{lstlisting}
%
Soluzione alternativa:

\begin{lstlisting}[style=Rstyle]
df[df$income_group_2017 %in% c("High income", "Low income"), ]
\end{lstlisting}
%
\noindent Punto 3: estrarre tutti i valori nella prima, seconda e ultima colonna

\begin{lstlisting}[style=Rstyle]
df[, c(1:2, 5)]
      country world_region gov_health_spending_percent
1       Benin       africa                        9.60
2      Greece       europe                       12.10
3    Tanzania       africa                       13.80
4     Estonia       europe                       11.70
5      Russia       europe                        8.04
6       Syria         asia                        5.58
7      Zambia       africa                       15.60
8     Vietnam         asia                        7.79
9     Liberia       africa                       11.10
10 Mozambique       africa                       12.20
\end{lstlisting}
%
Soluzione alternativa:

\begin{lstlisting}[style=Rstyle]
df[, c(1, 2, 5)]
\end{lstlisting}
%
\noindent Punto 4: estrarre tutti i valori tranne quelli dell'ultima colonna

\begin{lstlisting}[style=Rstyle]
df[, -5]
      country world_region income_group_2017
1       Benin       africa        Low income
2      Greece       europe       High income
3    Tanzania       africa        Low income
4     Estonia       europe       High income
5      Russia       europe     Middle income
6       Syria         asia     Middle income
7      Zambia       africa     Middle income
8     Vietnam         asia     Middle income
9     Liberia       africa        Low income
10 Mozambique       africa        Low income
   happiness_score_2011
1                  38.7
2                  53.7
3                  40.7
4                  54.9
5                  53.9
6                  40.4
7                  50.0
8                  57.7
9                    NA
10                 49.7
\end{lstlisting}
%
Soluzione alternativa:

\begin{lstlisting}[style=Rstyle]
df[, 1:4]
\end{lstlisting}
%
\noindent Punto 5: estrarre la prima, quarta, e quinta colonna per tutti i paesi africani

\begin{lstlisting}[style=Rstyle]
df[df$world_region == "africa", c(1,4,5)]	
      country happiness_score_2011
1       Benin                 38.7
3    Tanzania                 40.7
7      Zambia                 50.0
9     Liberia                   NA
10 Mozambique                 49.7
   gov_health_spending_percent
1                          9.6
3                         13.8
7                         15.6
9                         11.1
10                        12.2	
\end{lstlisting}

\end{solution}

\section{Capitolo 5}


\begin{solution}{ex4.1}

\begin{lstlisting}[style=Rstylescript]
# Calcolo la frequenza assoluta
freq_a_income <- table(gapminder$income_group_2017)

# Calcolo la frequenza relativa, 
freq_r_income <- prop.table(freq_a_income)

# Calcolo le percentuali moltiplicando per 100
freq_r_income <- freq_r_income*100

# Arrotondo a una cifra decimale
freq_r_income <- round(freq_r_income, 1)
\end{lstlisting}

\noindent La moda \`e: "Middle income"

\end{solution}

\vspace{0.5cm}

\begin{solution}{ex4.2}
	
\begin{lstlisting}[style=Rstylescript]
# Calcolo le percentuali moltiplicando per 100
tab_cont_rel_col_percent <- tab_cont_rel_col*100

# Arrotondo a una cifra decimale
tab_cont_rel_col_percent <- round(tab_cont_rel_col_percent, 1)
\end{lstlisting}

\noindent Le regioni al mondo pi\`u frequenti per low, middle e high income sono Africa, Asia e Europa, rispettivamente.

\end{solution}

\vspace{0.5cm}

\begin{solution}{ex4.3}
	
\begin{lstlisting}[style=Rstylescript]
#Calcolo il range come valore massimo meno valore minimo
range_happiness <- max_happiness - min_happiness

#Calcolo IQR come terzo quartile (in posizione 4 del vettore) - primo quartile (in posizione 2)
iqr_happiness <- quartile_happiness[4] - quartile_happiness[2]
\end{lstlisting}

\noindent Potrebbe avere una distribuzione asimmetrica a destra, ma non possiamo sapere se sia bi- o multi-modale.

\end{solution}

\vspace{0.5cm}

\begin{solution}{ex4.4}
	
\begin{lstlisting}[style=Rstylescript]
mean_spending <- mean(gapminder$gov_health_spending_percent)
sd_spending <- sd(gapminder$gov_health_spending_percent)

median_spending <- median(gapminder$gov_health_spending_percent, na.rm=TRUE)
quartile_spending <- quantile(gapminder$gov_health_spending_percent, na.rm=TRUE)

min_spending <- min(gapminder$gov_health_spending_percent, na.rm=TRUE)
max_spending <- max(gapminder$gov_health_spending_percent, na.rm=TRUE)

range_spending <- max_spending - min_spending
iqr_spending <- quartile_spending[4] - quartile_spending[2]

histogram_spending <- ggplot(gapminder, aes(x=gov_health_spending_percent)) + geom_histogram(binwidth=5, width=0.8, fill="white", color="red") + xlab("Score") + ylab("Counts") + ggtitle("Govt health spending in 2010 (% of total spending)", subtitle="Gapminder data") + theme_bw()
\end{lstlisting}

\noindent Per una spiegazione della soluzione di rimanda al testo delle Sezioni~\ref{sec:centdisp} e~\ref{sec:histo}

\end{solution}


\vspace{0.5cm}


\begin{solution}{ex4.5}

\begin{lstlisting}[style=Rstylescript]
boxplot_world_region <- ggplot(gapminder, aes(x=happiness_score_2011, y=world_region)) + geom_boxplot(fill="grey") + xlab("Score") + ylab("") + ggtitle("World happiness by world_region", subtitle="Gapminder data") + theme_bw()
\end{lstlisting}

\noindent Le persone sono meno felici in Africa.

\noindent Per una spiegazione della soluzione di rimanda al testo della Sezione~\ref{sec:boxplot}.

\vspace{0.2cm} 

\noindent Per risolvere l'ultimo punto, bisognava ricordarsi come estrarre parte di un data frame usando dei vettori (e quindi delle operazioni) logici, come introdotto in Sezione~\ref{sec:subsetting}.

\begin{lstlisting}[style=Rstylescript]
happiness_africa <- gapminder$happiness_score_2011[gapminder$world_region == "africa"]
mean(happiness_africa, na.rm=TRUE)

happiness_asia <- gapminder$happiness_score_2011[gapminder$world_region == "asia"]
mean(happiness_asia, na.rm=TRUE)

happiness_europe <- gapminder$happiness_score_2011[gapminder$world_region == "europe"]
mean(happiness_europe, na.rm=TRUE)

happiness_americas <- gapminder$happiness_score_2011[gapminder$world_region == "americas"]
mean(happiness_americas, na.rm=TRUE)
\end{lstlisting}


\end{solution}	

\newpage

\section{Capitolo 6}

\begin{solution}{ex5.1}

\noindent Le funzioni che abbiamo visto assumono come default una normale standardizzata.

\noindent Per rispondere alle domande, bisogna ricordare le propriet\`a della normale, in generale, e della normale standardizzata, in particolare.

\noindent La mediana (che corrisponde alla media, che nella normale standardizzata \`e zero) divide la curva in due parti identiche. Dato che l'area sottesa \`e sempre \lsin{1}, allora l'area a sinistra della mediana (\lsin{lower.tail = TRUE}) e alla sua destra (\lsin{lower.tail = FALSE}) sono entrambe \lsin{0.5}:

\begin{lstlisting}[style=Rstyle]
pnorm(0, mean = 0, sd = 1, lower.tail = TRUE)
[1] 0.5

pnorm(0, mean = 0, sd = 1, lower.tail = FALSE)
[1] 0.5
\end{lstlisting}
%
di conseguenza, il valore per cui si ottiene un'area uguale a \lsin{0.5} per una normale standardizzata \`e zero:

\begin{lstlisting}[style=Rstyle]
qnorm(0.5, lower.tail = TRUE)
[1] 0
\end{lstlisting}


\noindent Uno $z$-score di \lsin{1.96} \`e un valore particolare, che sappiamo corrisponde a un $\alpha$ di \lsin{0.05}, che viene diviso equamente tra la coda destra (1.96, \lsin{lower.tail = TRUE}) e la coda sinistra (-1.96, \lsin{lower.tail = TRUE}), ognuna delle quali vale quindi \lsin{0.025} (circa). Facendo la differenza, si ottiene l'area per .96, \lsin{lower.tail = FALSE}: \lsin{1-0.025=0.975} (circa):

\begin{lstlisting}[style=Rstyle]
pnorm(1.96, mean=0, sd = 1, lower.tail = TRUE)
[1] 0.9750021
pnorm(1.96, mean=0, sd = 1, lower.tail = FALSE)
[1] 0.0249979
 
pnorm(-1.96, mean=0, sd = 1, lower.tail = TRUE)
[1] 0.0249979
\end{lstlisting}
%
di conseguenza, il valore per cui si ottiene un'area uguale a \lsin{0.024} (coda sinistra) per una normale standardizzata \`e \lsin{-1.96} (circa):

\begin{lstlisting}[style=Rstyle]
qnorm(0.025, lower.tail = TRUE)
[1] -1.959964
\end{lstlisting}



\end{solution}

\vspace{0.5cm}	

\begin{solution}{ex5.2}

\noindent Possiamo usare direttamente la funzione \lsin{pnorm} con i parametri dell'esercizio, ricordandoci che vogliamo quelli che pesano meno di $1500 \text{ g}$, quindi la coda sinistra:

\begin{lstlisting}[style=Rstyle]
pnorm(1500, mean = 2404 , sd = 580, lower.tail = TRUE)
[1] 0.05954309
\end{lstlisting}

\noindent Possiamo anche andare a standardizzare il valore che ci interessa (ottenendo cos\`i lo $z$-score), che usiamo poi in \lsin{pnorm} assumendo che si tratti di una normale standardizzata (quindi con valori di default per media e deviazione standard):

\begin{lstlisting}[style=Rstyle]
zscore <- (1500-2404)/580
pnorm(zscore, lower.tail = TRUE)
[1] 0.05954309
\end{lstlisting}

\noindent Proporzione e probabilit\`a coincidono e sono uguali al $6\%$.

\end{solution}

\vspace{0.5cm}

\begin{solution}{ex5.3}

\noindent Sulla falsariga dell'esempio visto in Sezione~\ref{sec:CI}:

\begin{lstlisting}[style=Rstylescript]
# Stime conosciute
n <- 760
x <- 11.4
sd <- 11.2

# Calcolo lo standard error
se <- sd/sqrt(n)

# Calcolo il valore critico per il 90% CI e il corrispondente margine d'errore
z90 <- (1-0.90)/2
v90 <- qnorm(z90, lower.tail = FALSE)
me90 <- v90 * se

# Calcolo il margine superiore e inferiore:
lCI90 <- x - me90
uCI90 <- x + me90

# Calcolo il valore critico per il 99% CI e il corrispondente margine d'errore
z99 <- (1-0.99)/2
v99 <- qnorm(z99, lower.tail = FALSE)
me99 <- v99 * se

# Calcolo il	margine superiore e inferiore:
lCI99 <- x - me99
uCI99 <- x + me99
\end{lstlisting}


\begin{lstlisting}[style=Rstyle]
# Li visualizzo
lCI90
[1] 10.73175
uCI90
[1] 12.06825
 
lCI99
[1] 10.35353
uCI99
[1] 12.44647
\end{lstlisting}

\noindent L'intervallo di confidenza pi\`u largo \`e quello al 99\%, che ha pi\`u probabilit\`a di includere la vera media $\mu$ della popolazione, ma \`e anche meno preciso.

\end{solution}

\vspace{0.5cm}

\begin{solution}{ex5.4}
	
\noindent Definiamo per prima l'ipotesi nulla

\noindent $\mathcal{H}_0$: Non ci sono differenze in happiness score tra Europa e le Americhe

\begin{lstlisting}[style=Rstylescript]
# Seleziono le aree che mi interessano	
gapminder_eu_am <- gapminder[gapminder$world_region %in% c("europe", "americas"), ]

# Calcolo il $t$-test per un test a due code
r <- t.test(happiness_score_2011 ~ world_region, data=gapminder_eu_am, alternative = "two.sided")
\end{lstlisting}

\begin{lstlisting}[style=Rstyle]
# Visualizzo il P-value
r$p.value	
[1] 0.2485322
\end{lstlisting}
%
Quindi non possiamo rifiutare l'ipotesi nulla con una soglia critica $\alpha = 0.05$. Non abbiamo evidenze sufficienti che ci sia differenza tra Europa e le Americhe in termini di happiness score.
	
\end{solution}	


\vspace{0.5cm}

\begin{solution}{ex5.5}

\begin{lstlisting}[style=Rstylescript]
# Discretizzo
gapminder$spending_group <- ifelse(gapminder$gov_health_spending_percent < median_spending, "low", "high")

# Calcolo frequenza assoluta
spending_group <- table(gapminder$spending_group)
\end{lstlisting}


\begin{lstlisting}[style=Rstyle]
spending_group

high  low 
  91   89 
\end{lstlisting}

\end{solution}


\vspace{0.5cm}

\begin{solution}{ex5.6}
	
\noindent Definiamo per prima l'ipotesi nulla

\noindent $\mathcal{H}_0$: Non ci sono differenze tra world regions e avere una felicit\`a alta o bassa

\begin{lstlisting}[style=Rstylescript]
# Creo la tabella di contingenza	
tab_happiness_region <- table(gapminder$happiness_group, gapminder$world_region)
\end{lstlisting}


\begin{lstlisting}[style=Rstyle]
tab_happiness_region
       
        africa americas asia europe
  low       34        3   20     12
  right      4       19   20     27
\end{lstlisting}

\noindent Ci sono due celle con frequenza assoluta $< 5$, devo quindi usare il Fisher's test:


\begin{lstlisting}[style=Rstylescript]
rf2 <- fisher.test(tab_happiness_region)
\end{lstlisting}

\begin{lstlisting}[style=Rstyle]
# Visualizzo il P-value
rf2$p.value
[1] 9.054219e-10
\end{lstlisting}
%
Rifiutiamo l'ipotesi nulla con una soglia critica $\alpha = 0.05$ e concludiamo che ci siano differenze tra worls region e avere una felicit\`a alta o bassa.
	
\end{solution}	